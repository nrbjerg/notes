\chapter*{Results.}
\begin{definition}
Let $f: \R \to \R$ or $f: \R \to \C$, then it's called $2\pi$ periodic if $f(\theta + 2\pi) = f(\theta) \;\forall \theta \in \R$.
\end{definition}

\begin{definition}
  Fourier rækken skrives
  \begin{equation*}
    f(\theta) = \frac{a_{0}}{2} + \sum_{n=1}^{\infty} [a_{n}\cos(n \theta) + b_{n} \sin(n \theta)], \quad a_{i}, b_{i} \in \R
  \end{equation*}
  Skrives på komplex form som
  \begin{equation*}
    f(\theta) = \sum_{n=-\infty}^{\infty} c_{n} \e^{in \theta}
  \end{equation*}
  hvor $c_{0} = \frac{a_{0}}{2}$ og $c_{n} = \frac{1}{2} (a_{n} + sign(n) i b_{n})$ for $n \in \Z$
\end{definition}

\begin{remark}
  koefficienterne har formen
  \begin{equation*}
    c_{k} = \frac{1}{2\pi} \int_{-pi}^{pi} f(\theta) e^{-ik \theta} d\theta
  \end{equation*}
  og tilsvarende
  \begin{align*}
    a_{n} &= \frac{1}{\pi} \int_{-\pi}^{\pi} f(\theta) \cos(n \theta) d\theta \\
    b_{n} &= \frac{1}{\pi} \int_{-\pi}^{\pi} f(\theta) \sin(n \theta) d\theta
  \end{align*}
  for $n \in \N$
  Og såfremt $\phi_{n}$ definieres som $\frac{1}{\sqrt{2\pi}} \e^{in x}$ er
  \begin{equation*}
    c_{n} = \frac{1}{\sqrt{2\pi}} \gen{f, \phi_{n}}
  \end{equation*}
\end{remark}

\begin{definition}
  En $2\pi$ periodisk funktion $f$ kaldes intergrabel hvis \begin{equation*}
    \int_{-\pi}^{\pi} \abs{f(\theta)} d \theta < \infty
  \end{equation*}
\end{definition}
for intergrable $f$ er
\begin{equation*}
  \int_{-\pi}^{\pi} f(\theta) \e^{-ik \theta} d \theta
\end{equation*}
veldefinieret for alle $k \in \Z$.

\begin{remark}
  Hvis $g$ er ulige st. $g(\theta) = -g(-\theta)$, så er
  \begin{equation*}
    b_{n} = 0
  \end{equation*}
  for $n \in \N$ ellers hvis $g$ lige så er
  \begin{equation*}
    a_{n} = 0
  \end{equation*}
  for $n \in \N$
\end{remark}

\begin{theorem}[Bessel's ulighed]
  lad $f$ være $2\pi$ periodisk, og Riemann integrable, så gælder
  \begin{equation*}
    \sum_{n=-\infty}^{\infty} \abs{c_{n}}^{2} \leq \frac{1}{2\pi} \int_{-\pi}^{pi} \abs{f(\theta)}^{2} d \theta
  \end{equation*}
\end{theorem}

\begin{lemma}[Reimann-Lebesgues]
  $f$ Riemann integrable og $2\pi$ periodisk $\implies$
  \begin{equation*}
    \lim_{n \to \pm \infty} c_{n} = 0
  \end{equation*}
\end{lemma}

\begin{definition}
En funktion $f: [a, b] \to \C$ kaldes \textbf{stykvis kont.} hvis $f$ er kont. på $[a, b]$ untagen endeligt mange punkter. $x_{1}, x_{2}, x_{k}$ hvor grænserne for højre og venstre eksister. $f$ kaldes \textbf{stykvis glat} hvis $f$ og $f'$ er stykvis kont. på $[a, b]$
\end{definition}

\begin{definition}
  Partial summen af orden $N$ definieres som:
  \begin{equation*}
    S_{N}^{f} (\theta) = \sum_{n=-N}^N c_{n} \e^{in \theta}
  \end{equation*}
\end{definition}

\begin{remark}
  Det bemærkes at
  \begin{equation*}
    S_{N}^{f}(\theta) = \int_{-\pi}^{\pi} f(\theta + \phi) D_{N}(\phi) d \phi
  \end{equation*}
  hvor $D_{N} = \frac{1}{2\pi} \sum_{n=-N}^N \e^{in \theta}$ er den såkaldte \textbf{dirichlet kerne}.
\end{remark}

\begin{theorem}
  Lad $f$ være $2\pi$ periodisk og stykvis glat. så gælder det at
  \begin{equation*}
    \lim_{n \to \infty} S_{n}^{f} (\theta) = \frac{1}{2} [f(\theta-) + f(\theta+)], \theta \in [-\pi, \pi]
  \end{equation*}
  specielt gælder det at $f$ kont i $\theta$ medføre at:
  \begin{equation*}
    \lim_{n \to \infty}  S_{n}^{f} (\theta) = f(\theta)
  \end{equation*}
\end{theorem}

\begin{proposition}
  Lad $f$ kont. og stykvis glat på $[-\pi, \pi]$, så gælder det at
  \begin{equation*}
    c'_{n} = in c_{n}
  \end{equation*}
  hvilket giver os en mulighed for at beregne kvotienterne for den afledte fourier række
\end{proposition}
\begin{remark}
  Derudover giver dette resultat af fourier rækken for $f$ konvergere både absolut og uniformt i dette tilfælde på $[-\pi, \pi]$.
\end{remark}

\begin{proposition}
  Cauchy-Schwarz ulighed giver at for to reelle følger $\{a_{n}\}$ og $\{b_{n}\}$ gælder at
  \begin{equation*}
    \sum_n a_{n} b_{n} \leq \left(\sum_n a_{n}^{2} \right)^{\frac{1}{2}} \left(\sum_n b_{n}^{2}\right)^{\frac{1}{2}}
  \end{equation*}
\end{proposition}
\begin{proposition}[Weierstrass' M-test]
  lad $f_{n}: A \to \C$ for $n \in \Z$. Hvis der findes en følge $M_{n} \in (0, \infty)$ st. $|f_{n}(x)| \leq M_{n}$ for $x \in A$, $n \in \Z$ og $\sum_{n \in \Z} M_{n} \leq \infty$, så konvergere
  \begin{equation*}
    \sum_{n\in \Z} f_{n}(x)
  \end{equation*}
  uniformt på $A$.
\end{proposition}

% SELF STUDY 1 GOES HERE

\begin{theorem}
  Suppose $f$ is $2\pi$ periodic and peicewise continues, with forier coef. $a_{n}, b_{n} c_{n}$ and let $F(\theta) = \int_0^{\theta} f(\phi) d \phi$. If $c_{0} = 0$ (this is the same as $a_{0} = 0$), then $\;\forall \theta$ we have
  \begin{equation*}
    F(\theta) = C_{0} + \sum_{n \neq 0} \frac{c_{n}}{in} \e^{in \theta} = C_{0} + \sum_{n=1}^{\infty} \left(\frac{a_{n}}{n} \sin(n \theta) - \frac{b_{n}}{n} \cos(n \theta)\right)
  \end{equation*}
  Where
  \begin{equation*}
    C_{0} = \frac{1}{2\pi} \int_{-\pi}^{\pi} F(\theta) d \theta
  \end{equation*}
\end{theorem}

\begin{definition}
\textbf{Gibbs phenomenon}: as one adds more and more terms the paritals shums overshoot and undershoot $f$ near the discontinuity.
\end{definition}

\begin{definition}
\begin{equation*}
  L^{2}(a, b) = \{f \int_a^b \abs{f(x)}^{2} dx < \infty\}
\end{equation*}
hvor intergralet er det såkaldte lebesgue intergrale.
\end{definition}
\begin{theorem}
$L^{2}(a, b)$ er et hilbertrum (fuldstændigt normeret vektorum)
\end{theorem}

\begin{theorem}[Sætningen om domineret konvergens]
Lad $D \subseteq \R^{k}$ være et område. Antag, at $\{g_{n}\}_{n \in \N}$, er en følge af funktioner på $D$ og at $phi, g$ er funktioner på $D$, således
\begin{enumerate}[i)]
  \item $\phi(\mathbf{x}) \geq 0$ og $\int_{D} \phi(\mathbf{x}) d \mathbf{x} < \infty$ (Her er $d \mathbf{x}$ det såkaldte lebesgue mål)
  \item $\abs{g_{n}(\mathbf{x})} \leq \phi(\mathbf{x})$, for alle $\mathbf{x} \in D, n \in \N$.
  \item $g_{n}(\mathbf{x}) \to g(\mathbf{x})$ for alle $\mathbf{x} \in D$.
\end{enumerate}
så gælder det at
\begin{equation*}
  \int_{D} g_{n}(\mathbf{x}) d \mathbf{x} \to \int_{D} g(\mathbf{x}) d \mathbf{x}
\end{equation*}
\end{theorem}

\begin{definition}
  Følgende vektorrum spiller en stor rolle:
  \begin{align*}
    L^{1}(\Z) &= \{x = \{x_{n}\}_{n \in \Z} \mid \norm{x}_{1} := \sum_{n\in \Z} \abs{x_{n}} < \infty \}\\
    L^{1}(\Z) &= \{x = \{x_{n}\}_{n \in \Z} \mid \norm{x}_{2} := \left(\sum_{n\in \Z} \abs{x_{n}}^{2}\right)^{1/2} < \infty \}\\
    L^{\infty}(\Z) &= \{x = \{x_{n}\}_{n \in \Z} \mid \norm{x}_{\infty} := \sup_{n} \abs{x_{n}} < \infty \}
  \end{align*}
\end{definition}
\begin{remark}
  Det kan vises at $L^{1}(\Z) \subset L^{2}(\Z) \subset L^{\infty}(\Z)$, bemærk at det her er over $\Z$! og at $L^{2}(\Z)$ er et hilbertrum.
\end{remark}

\begin{definition}
  Kroneckers delta følge:
  \begin{equation*}
    \delta_{n} = \begin{cases} 1, & n = 0 \\ 0, & n \neq 0 \end{cases}
  \end{equation*}
\end{definition}

\begin{definition}
  Lad $y = T(x)$, hvor $T: V \to V$ være et diskrettids system.
  \begin{itemize}
    \item Systemet kaldes lineært hvis $T(\alpha x + \beta y) = \alpha T(x) + \beta T(y)$ for alle $\alpha, \beta \in \C, x, y \in V$
    \item Systemet kaldes tidsinvariant, hvis
      \begin{equation*}
        y' = T(x') \text{ hvor } \begin{cases}
                                   x'_{n}= x_{n - k}
                                   y'_{n}= y_{n - k}
                                 \end{cases}
      \end{equation*}
      for alle $k \in \Z$
    \item Systemet kaldes BIBO-stabilt (bounded-in, bounded-out) hvis
      \begin{equation*}
        x \in L^{\infty}(\Z) \implies y = T(x) \in L^{\infty}(\Z)
      \end{equation*}
    \item Systemet kaldes hukommelsesløst hvis for $x, x'$ gælder at
      \begin{equation*}
        x_{k} = x_{k}' \implies T(x)_{k} = T(x')_{k}
      \end{equation*}
    \item Systemet kaldes kausal hvis det for $x, x'$ og $k \in \Z$ gælder at:
        \begin{equation*}
          1_{-\infty, \ldots, k}x = 1_{-\infty, \ldots, k}x' \implies 1_{-\infty, \ldots, k}T(x) = 1_{-\infty, \ldots, k}T(x')
        \end{equation*}
      hvor $(1_{-\infty, \ldots, k}x)_{l} = \begin{cases}
                                              x_{l}, & l\leq k\\
                                              0, & l > k
                                            \end{cases}$
      for $l \in \Z$
  \end{itemize}
\end{definition}

\begin{definition}
Lad $T: V \to V$ være både lineært og tidsinvariant, så definieres impulsresponsen som $h = T(\delta)$.
\end{definition}
\begin{remark}
  Vi har
  \begin{equation*}
    y = T(x) =  T\left(\sum_{k\in \Z} x_{k} \delta_{n - k}\right) = \sum_{k\in \Z} x_{k} T(\delta_{n - k}) = \sum_{k\in\Z} x_{k} h_{n -k} = h * x
  \end{equation*}
\end{remark}
