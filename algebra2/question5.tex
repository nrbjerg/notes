\chapter*{5. Cyclotomic polynomials and roots of unity}
\begin{lemma}
  $\xi \in \C$ is a primitive $n$'th root of unity iff
  \begin{equation*}
    \xi = \e^{2 k\pi i / n},
  \end{equation*}
  st. $1 \leq k \leq n$ and $gcd(k, n) = 1$. If $\xi$ is a primitive $n$'th root of unity and $\xi^{m} = 1$ then $n | m$.
\end{lemma}

\begin{theorem}
  Let R be a domain and $f \in R[X]\setminus \{0\}.$ If $V(f) = \{\alpha_{1}, \alpha_{2}, \ldots, \alpha_{r}\}$ then.
  \begin{equation*}
    f = q \prod_{k = 1}^{r} (X - \alpha_{k})^{v_{\alpha_{k}}(f)}
  \end{equation*}
  where $q \in R[X]$ and $V(q) = \emptyset$. And $\sum^{r}_{k = 1} v_{\alpha_{k}}(f) \leq deg(f)$.
\end{theorem}

\begin{proposition}
  Let $n \in \N$. Let $f = X^{n} - 1$ and $g = \displaystyle\prod_{d | n} \Phi_{d}(X)$ then $f = g$
\end{proposition}
\begin{proof}\
  \begin{enumerate}[i)]
    \item Both $f$ and $g$ are monic $\therefore$, they are equal if they both have the same roots, with the same multiplicites, this follows from theorem. Since $f \text{ monic }\implies q = 1$ in the thm.
    \item We will now proof that $f$ and $g$ have the same roots with the same multiplicites. The roots of $g$ is the $d$'th primitive roots of unity st. $d | n$, these are also roots of $f$: Let $x$ be a root of $g$ then
    \begin{equation*}
      d | n \implies \exists k \in \Z \text{ st. } d\cdot k = n \implies x^{n} = x^{d \cdot k} = 1^{k} = 1.
    \end{equation*}
    Now $\xi$ begin a root of $f$ means $\xi$ is a $n$'th root of unity, however this implies that $\xi$ is a $d$'th primitive root of unity, where $d | n$, consult the lemma. Thus $\xi$ is also a root of $g$. Lastly none of the roots have multiplicty higher than one.
  \end{enumerate}
\end{proof}
